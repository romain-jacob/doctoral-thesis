% !TEX root = ../00_thesis.tex

\section{Summary}
\label{sec:ttw_conclusion}

%What we presented
In this chapter, we presented Time-Triggered Wireless (\TTW), a time-triggered design for wireless \CPS.
\TTW provides end-to-end real-time guarantees by statically co-scheduling all tasks and messages in the system, which is performed offline by resolving a MILP formulation.
This approach is inspired by similar work in the wired domain, in particular the real-time scheduling of FlexRay buses.
Compared to \DRP~(\cref{ch:drp}), \TTW's static schedules allow to meet shorter end-to-end deadlines \feature{Efficiency} at the cost of a lesser (\feature{Adaptability}); indeed, \TTW's runtime adaptability is limited to switching between pre-defined operation modes.

% Key concept/novel idea
The main challenge in the \TTW design is that, with wireless communication, it is highly beneficial in terms of energy to send messages in rounds. Thus, the assignment of messages to round (similar to a bin-packing problem) must be combined to the traditional co-scheduling approaches, which is non-trivial.

% Take aways
We solved this problem and implemented a multi-mode scheduler that allows critical applications to seamlessly switch between modes while minimizing the energy consumption spent for wireless communication.
We further implemented a predictable network stack, called \TTnet. Together, these two pieces from \TTW, a publicly available~(\cref{appendix:ttw_artifacts}) real-time wireless \CPS design.

% !TEX root = ../00_thesis.tex

\section{Related Work}
\label{sec:relwork}

As mentioned in the introduction, the idea of middleware for Wireless Sensor Networks (WSN) has been around for more than a decade~\cite{romer2004Programming,chatzigiannakis200750,wang2008Middleware,mottola2012Middleware}.
These papers generally agree on the needs and challenges for WSN middlewares.
Yet, there have been relatively few proposals to address these challenges.
Recent surveys~\cite{razzaque2016Middleware,onderwater2016overview} provide an overview of the middleware literature in the wider context of the Internet of Things, which covers all layers from local devices to cloud services.
\cite{mottola2012Middleware} reviewed the literature focusing more on WSN:
proposals include for example Impala~\cite{liu2003Impala} which explicitly address the problem of fault tolerance in mobile networks.
Programming abstractions like TinyDB~\cite{madden2005TinyDB}, RUNES~\cite{costa2005RUNES}, or TinyLIME~\cite{costa2007Programming} have also been proposed.
However, in these works, the level of abstraction is either higher or lower than the network layer.

In the past two decades, countless wireless MAC protocols have been proposed (see \eg \cite{teleshermeto2017Scheduling,bartolomeu2016Survey} for recent surveys).
%
\cite{isolani2018Survey} surveyed and classified Wireless MAC protocols according to their programmability \emph{scope} (what elements of the MAC layer are programmable) and \emph{level} (the granularity at which the protocol logic can be programmed). The authors classify protocols as either monolithic, parametric or modular.
In the context of IEEE 802.15.4 networks, modular protocols include \eg the MAC Layer Architecture (MLA)~\cite{klues2007MLA} and $\lambda$-MAC~\cite{parker2010lambda}. In contrast, \baloo would be classified as parametric, as it allows ``parameter tuning through interfaces''.
Using Synchronous Transmissions (\ST) interestingly changes the way network stacks can or should be designed.
So far, only few network stacks using \ST have been proposed~\cite{ferrari2012LWB,istomin2018Interferenceresilient,sarkar2016Sleeping,sutton2017eLWB,jacob2017TTW_extended,suzuki2013Choco}, and almost no research has been conducted to propose a flexible design framework (\eg comparable to MLA~\cite{klues2007MLA}) but tailored to \ST.
Two noteworthy exceptions are A$^2$~\cite{alnahas2017a2} and Atomic-SDN~\cite{baddeley2019AtomicSDN}.

% A$^2$, a work with a very similar approach to ours.
A$^2$ aims to facilitate the design of \ST-based communication using a middleware component, called Synchrotron. However, A$^2$ is not a network stack, it is a \emph{generic \ST primitive}.
\baloo and A$^2$ actually complement each other perfectly:
\baloo facilitates the design of network layer protocols, but it requires to have \ST primitives (\eg Glossy~\cite{ferrari2011Glossy} or Chaos~\cite{landsiedel2013Chaos}) available, which are typically hard to implement.
In turn, A$^2$ facilitates the design of such \ST primitives.
The support of A$^2$ within \baloo would be a natural next step towards a fully flexible and configurable network stack based on \ST.

Atomic-SDN~\cite{baddeley2019AtomicSDN} is another work sharing this idea of a fully configurable network stack. \baloo and Atomic-SDN are very similar pieces of software, which have been developed in parallel.
Compared to \baloo, Atomic-SDN is slightly more specialized: it pre-defines top-level functions (collection, configuration, reaction, and association) with a given implementation, which the user can schedule on-demand. By contrast, \baloo leaves the user  access the various \ST primitives and compose them freely to implement the desired protocol logic.
Another key difference is precisely that \baloo lets the user pick-choose-and-combine the primitives to use, whereas Atomic-SDN is restricted to only one primitive (in the current implementation). Atomic-SDN uses a back-to-back transmissions schemes (\cite{baddeley2019Competition, lim2017Competition}) instead of traditional Glossy floods~\cite{ferrari2011Glossy}.

% !TEX root = ../00_thesis.tex

\section{Summary}

%What we presented
This chapter presented \baloo, a flexible design framework for low-power wireless network stacks based on \ST.
We illustrated its \feature{Usability} and \feature{Generality} by re-implementing three well-known network stacks: the Low-power Wireless Bus (LWB)~\cite{ferrari2012LWB}, Sleeping Beauty~\cite{sarkar2016Sleeping}, and Crystal~\cite{istomin2018Interferenceresilient}, and we showed that using \baloo induces only limited performance overhead in terms of radio duty cycle and memory usage.
\baloo supports the use of multiple \ST primitives within the same network stack (\feature{Versatility}) while guaranteeing that the timing requirements for \ST are met (\feature{Synchronicity}).

% Key concept/novel idea
The key concept of \baloo is its clean API, based on callback functions, which let the users focus on implementing the protocol logic without worrying about low-level radio control (interrupt handling, timer settings, \etc).
The API is generic and supports the different communication primitives. Through this API, multiple primitives can be used within the same network stack without additional complexity for the users.

% Take aways
The code of \baloo is openly available and is accompanied by a detailed documentation of its features and how to use them~(\cref{append:baloo_artifacts}).
Our re-implementations of Crystal, Sleeping Beauty, and LWB are also available.
We believe \baloo will be an important enabler for the development of future real-world applications leveraging state-of-the-art \ST technology.

% !TEX root = 00_thesis.tex
\chapter[Abstract]{Abstract}

% Context
%% CPS
\startsquarepar
Cyber-Physical Systems (\CPS) refer to systems where some intelligence is embedded into devices that interact with their environment; that is, collecting information from the physical space, processing that information, and taking actions that affect the environment. Automatically turning the heating on when room temperature gets cold is one of the simplest example of \CPS.\linebreak
%
Things get more complex when applications are distributed between low-power devices that should operate autonomously for multiple years. Then, performing reliable and energy efficient wireless communication becomes paramount.
%
Moreover, applications often specify deadlines; that is, maximal tolerable delays between the execution of distributed tasks.
Systems that guarantee to meet such deadlines are called real-time systems.
%
Wireless \CPS capable of providing real-time guarantees while using low-power communication technology are desirable but they are particularly challenging to design.
%% ST
In the past few years, a technique known as synchronous transmissions (\ST) has been shown to enable reliable and energy efficient communication in low-power multi-hop networks.
\linebreak
In a nutshell, \ST consists in letting multiple devices transmit a packet during the same time interval; communication is likely to be successful if the transmissions are well synchronized, hence the name of \emph{synchronous} transmissions.
\ST can be leveraged to realize any multi-hop broadcast --~a one-to-all communication~-- in a given time; a very interesting property for designing real-time systems.
\stopsquarepar

% Need: what you have vs. what you want
While the potential of \ST is recognized by the low-power wireless academic community, this technique has not yet been leveraged for the design of \CPS.\linebreak
We identify at least three issues that limit the adoption of \ST in this domain:
\linebreak
\smallBox{(i)}\ST is difficult to use due to stringent time synchronization requirements: in the order of \us.
There is a lack of tools to facilitate the implementation of \ST by \CPS engineers, which are often not wireless communication experts.
\linebreak
\smallBox{(ii)}There are only few examples showcasing the use of \ST for \CPS applications and academic works based on \ST tend to focus on communication rather than applications. Convincing proof-of-concept \CPS applications are missing.
\linebreak
\smallBox{(iii)}The inherent variability of the wireless environment makes performance evaluation challenging. The lack of an agreed-upon methodology hinders experiment reproduciblility and limits the confidence in the performance claims.

% task
Consequently, we developed support tools and methods to facilitate the evaluation of wireless protocols and the implementation of \CPS based on \ST.
\linebreak
Furthermore, we leveraged \ST to design two \CPS solutions targeting different classes of real-time applications.
% object
This dissertation presents these contributions.

\begin{itemize}

  % =============================== %
  \item
  \startsquarepar
  In \cref{ch:triscale}, we propose to design and analyze performance evaluation experiments for networking protocols using a concrete, rational, and statistically sound methodology.
  We implement this methodology in a framework called \triscale  which allows to make performance claims with quantifiable levels of confidence.
  Furthermore, we leverage the \triscale framework to propose the first formalized definition of reproducibility for networking experiments.
  \stopsquarepar

  % =============================== %
  \item
  \startsquarepar
  \cref{ch:baloo} presents \baloo, a flexible design framework for network stacks based on \ST.
  Users implement their protocol through the programming interface offered by \baloo while the framework handles the complex low-level operations; \eg meeting the time synchronization requirements of \ST.
  \linebreak
  We show that \baloo is flexible enough to implement a wide variety of communication protocols while introducing only limited memory and energy overhead.
  \stopsquarepar

  % =============================== %
  \item
  Finally, we design and implement two wireless \CPS based on \ST:
  {\setlength{\parskip}{0pt}%
    \begin{itemize}[nosep]
      \item the Distributed Real-time Protocol (\DRP) uses contracts to maximize the flexibility of execution between distributed tasks~(\cref{ch:drp});
      \item Time-Triggered Wireless (\TTW) statically co-schedules all task executions and packet transfers to minimize end-to-end latency~(\cref{ch:ttw}).
    \end{itemize}
    \startsquarepar
    % Findings
    We demonstrate that real-time guarantees can be provided in a reliable and energy efficient manner.
    Furthermore, \TTW supports update rates of tens of \ms, which is sufficient to perform distributed closed-loop control of inverted pendulums -- a fundamental benchmark for control and robotic applications.
    \stopsquarepar
  }

\end{itemize}

% Conclusion
\startsquarepar
With this dissertation, we showcase that \ST is suitable to meet the requirements of real-time wireless \CPS. Furthermore, we facilitate the implementation of such systems with \baloo, a design framework that makes \ST accessible to the non-expert. Finally, \triscale provides an important building block to confidently evaluate the performance of networking protocols -- an essential building block of wireless \CPS.
% Perspectives
Building on \triscale, it would be useful to define benchmark problems representative of different classes of applications to serve as baseline for the evaluation of future wireless \CPS solutions. Ultimately, we must transition from proof-of-concepts to real-world wireless \CPS applications; this would be further facilitated by porting \baloo to newer and more powerful platforms, thereby pushing the limits of achievable performance levels.
\stopsquarepar

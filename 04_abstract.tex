% !TEX root = 00_thesis.tex

\chapter[Abstract]{Abstract}

% Context 1: information required to understand the need

% Need 1: what you have vs. what you want
One of the foundations of science is that a result can be considered valid only if it can be reproduced by others.
In low-power wireless research however, assessing reproducibility remains a grand challenge, in part due to the intrinsic variability  of the wireless environment.
Without reproducibility, one cannot hope to compare different results.

The field of low-power wireless networking has tremendously progressed over the past decade:
Researchers regularly report impressive performance results with respect to reliability (packet reception rate of 99.9\% and above), latency (end-to-end latency the order of milliseconds), and energy consumption (autonomous life time of multiple years).
With such performance levels, it becomes all the more important to enable sound comparisons between different protocols and approaches.

% Context 2
Some of the performance improvements we mentioned were enabled by an approach called synchronous transmissions (\ST).
% st is a technique that does xyz.
In a nutshell, \ST consists in letting multiple nodes transmit a packet during the same time interval; communication is likely to be successful (\ie nodes in the vicinity receive a packet) if the transmissions are sufficiently well synchronized, henceforth the name of \textit{synchronous} transmissions.
\ST has been shown to be very reliable and efficient in multi-hop networks.
%, with the potential of providing new services for Internet of Things (IoT) or cyber-physical systems applications.

% Need 2: what you have vs. what you want
Unfortunately, the usability of \ST is limited by its very high sensitivity to precise timing. This sensitivity makes \ST-based protocols difficult to interface with the higher layers of a communication stack.
Consequently, the adoption of \ST is currently limited to a relatively small circle of academic practitioners, despite almost a decade of development and multiple demonstrations of its potential for real-world applications.

% Task: what we did to address the need
In this dissertation, we first address the general challenge of repeatability and comparability in low-power wireless networking.
We then focus on the \ST approach:
we strive to make \ST more accessible and we demonstrate how it can efficiently provide certain performance guarantees in distributed applications.
Specifically, we make the following contributions:

\begin{description}

% =============================== %
% Contribution 1
% =============================== %
%
\item[\chref{ch:triscale}]%
% Object
We study the process of evaluating and comparing the performance of low-power networking protocols.
% Findings
We propose a statistically sound methodology to assess the repeatability of experiments, tailored to the context of wireless networking,
and we formalize this methodology as a set of guidelines.
% Conclusions
Following this methodology allows to make provable claims about a protocol performance, but also to compare protocols; for example, making (or refuting) claims such as \emph{``protocol A is better than protocol B''}.

% =============================== %
% Contribution 2
% =============================== %

\item[\chref{ch:baloo}]%
% Object
We propose \baloo, a flexible design framework for network stacks based on synchronous transmissions (\ST).
% Findings
We show that \baloo is flexible enough to implement a wide variety of network layer protocols, while introducing only limited memory and energy overhead. Most importantly \baloo makes \ST accessible: The framework is open source, well documented,
% Conclusions
and has already been used and extended by independent research groups.\rj{For now, such developments are plans... we shall see what happens in the next months.}

% =============================== %
% Contribution 3
% =============================== %
%
\item[\chref{ch:realtime}]%
% Object
We design and implement two different network stacks that leverage \ST to efficiently provide end-to-end real-time guarantees in a multi-hop wireless network:
% Findings
the Distributed Real-time Protocol (\DRP) offers flexibility between the execution of distributed tasks;
Time-Triggered Wireless (\TTW) targets minimal end-to-end latency.
% Findings
Both protocols have different design goals while providing end-to-end real-time guarantees.
% Conclusion
We illustrate the respective performance of these network stacks and evaluate how they compare against each other using the methodology we introduce in [\chref{ch:triscale}].
% =============================== %

\end{description}

% Perspectives

% -> These are not perspectives. It is merely a reformulation of the object (this paragraph could replace the one before the contributions...)

%This work addresses some important challenges of state-of-the-art low-power wireless technology:
%We contribute to make the \ST approach more accessible,
%we illustrate how it enables novel designs with unprecedented performance guarantees, and
%we tackle the need for repeatability and comparability in low-power wireless networking, which, beyond \ST, applies to any wireless communication approach.

\newpage
\section*{Claims}
\begin{itemize}

  \item
  We work towards more rigorous and reproducible experimental networking research.
  For the first time, we go beyond simple guidelines and propose a concrete methodology for designing networking experiments and analyzing their data.
  We leverage this methodology to propose the first formalized definition of reproducibility for networking experiments.
  We implemented our methodology in a framework called \triscale, a first-of-its-kind tool that assists researchers by streamlining the design process and automating the data analysis.

  \item
  We propose and implement \baloo, a design framework for network stacks based on synchronous transmissions.
  \baloo significantly lowers the entry barrier for harnessing the efficiency, reliability and mobility support of synchronous transmissions:
  users implement their protocol through a simple yet flexible API while \baloo handles all the complex low-level operations based on the users' inputs.

  \item
  We demonstrate for the first time that end-to-end real-time guarantees can be obtained in low-power wireless networks by leveraging the efficiency and reliability of synchronous transmissions.
  We propose and implement wireless real-time protocols for two different design objectives.
  \begin{itemize}
    \item Distributed Real-time Protocol (\DRP) uses contracts to maximize the flexibility of execution between distributed tasks.
    \item Time-Triggered Wireless (\TTW) statically co-schedules all task executions and message transfers to minimize end-to-end latency.
  \end{itemize}

\end{itemize}

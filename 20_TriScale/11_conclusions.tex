% !TEX root = ../00_thesis.tex

%-------------------------------------------------------------------------------
\section{Summary}

%What we presented
Establishing a consistent methodology for the design of networking experiments and the analysis of their data is a crucial step towards a more rigorous and reproducible scientific activity.
This chapter presented \triscale, the first concrete proposal in that direction.

% Key concept/novel idea
\triscale implements a methodology grounded on non-parametric statistics into a framework that assists scientists in designing experiments and automating the data analysis.
\triscale ultimately improves the legibility of results and helps quantifying the reproducibility of experiments, as highlighted in the case studies presented throughout the chapter.
% Takeaways
We expect \triscale's open availability to actively encourage its use by the networking community and promote better experimentation practices in the short term.
The quest towards highly-reproducible networking experiments remains open, but we believe that \triscale represents an important stepping stone towards an accepted standard for experimental evaluations in networking.
